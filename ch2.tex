%% for  TeXShop
% !TEX encoding = UTF-8 Unicode
% !TEX root = fermi.tex
%% for TeXShop

\chapter*{II. 熱力学第一法則}
\addcontentsline{toc}{chapter}{II. 熱力学第一法則}
\section*{3. 熱力学第一法則の内容}
\addcontentsline{toc}{section}{3. 熱力学第一法則の内容}
熱力学第一法則は本質的には熱力学系のエネルギー保存則である.
そのように考えると, 熱力学第一法則は, どのような熱力学的変化においても, 系のエネルギーの変化は系が環境から受け取るエネルギーの量と等しい, と表現できる.
この statement の意味を正確にするため, 「系のエネルギー」と「熱力学的変化で系が環境から受け取るエネルギー」という言葉を定義する必要がある.\par
純粋に力学的な保存系では、エネルギーはポテンシャルエネルギーと運動エネルギーの和であるため、系の力学的状態の関数である (系の力学的状態を知ることは系に含まれる質点の位置と速度を知ることに等しいため)。
系に働く外力がなければ、エネルギーは一定である。
つまり、$A$と$B$が孤立系の2つの連続した状態とし、 $U_A$, $U_B$を対応するエネルギーとすると、

\begin{align}
    U_A = U_B.
\end{align}
