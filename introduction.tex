% !TEX root = fermi.tex
% !TEX encoding = UTF-8 Unicode
%% for TeXShop
\chapter{Introduction}
熱力学は主に、熱の力学的仕事への変換と、その逆の力学的仕事の熱への変換を扱う. \par
かなり最近になってから, 物理学者達は熱はエネルギーのひとつの形態であり, 他のエネルギーの形態に変換できることを知った. かつては, 科学者達は熱は流体のようなもので, その送料は変わらないと考えられており, 流体がある物体から別の物体に流れていくように, 熱もまた物体の間を移動するものとされた. したがって, カルノー (Carnot) が1824年に, 熱を仕事へ変換するときに掛かる制限について, つまり, 今日で言う熱力学第2法則に対して, 熱-流体理論 (heat-fluid theory) \footnote{訳注: 熱波動説?} に基づいて比較的明快な理解に辿り着いたことは特筆すべきことである (第3章を参照). \par
わずか18年後の1842年, メイヤー (R. J. Mayer) は熱と力学的仕事の等価性を発見し, エネルギー保存の原理について初めて言及した (熱力学第1法則). \par
今日, 我々は熱と力学的エネルギーの等価性の実際の根幹, つまり, 全ての熱的な現象は原子や分子の運動に還元できるということを知っている. このことから, 熱学は力学の一分野だと考えられる. それは膨大な数の粒子 (原子や分子) の集合 (アンサンブル) の力学である. ここでは個々の粒子の状態や運動の詳細な描写は重要性を失い, 多数の粒子の平均的な性質だけが考慮される. このような力学の分野を\textbf{統計力学} (statistical mechanics) と呼び, 主にマクスウェル (Maxwell) やボルツマン (Boltzmann), ギブズ (Gibbs) らの仕事によって発展してきた. 統計力学によって, 熱力学の法則は根本的に理解されるようになった. \par
一方で純粋な熱力学のアプローチはこれとは異なり, 基本法則は実験事実から仮定され, 結論は現象の運動論には立ち入らずに導かれる. この手法は, 統計力学的な考察でよく行われるような単純化する仮定を行わなくてよいという利点がある. よって, 熱力学の結果は一般にとても正確である. しかし, 物事がどう動作するか詳しく見ずに結果を得ることは不十分なことがあるので, 多くの場合, 最低でもおおまかな力学的解釈で熱力学的結果を補完するとよい. \par
熱力学の第1, 第2法則は古典力学に統計的土台を持つ. 近年, ネルンスト (Nernst) が第3法則を追加したが, それは量子力学的にしか統計的解釈ができない. この本の最後の章で第3法則について扱う. 

